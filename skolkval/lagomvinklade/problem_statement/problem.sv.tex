\problemname{Lagomvinklade trianglar}
En \emph{lagomvinklad} triangel är vad vi i denna uppgift kallar en triangel där minst en av vinklarna är exakt 60 grader. De lagomvinklade trianglarna känner sig ofta förbisedda jämfört med de mycket mer kända rätvinkliga trianglarna (så kallat mindervinkelkomplex), trots att de lagomvinklade också har en snygg formel för sina sidlängder:
\begin{equation*}
	c^2 = a^2 + b^2 - ab
\end{equation*}
Skriv ett program som skipar lite rättvisa i detta triangeldrama genom att fråga efter ett tal $N$ (mellan $1$ och $100$) och sedan skriva ut hur många lagomvinklade trianglar det finns vars sidor är heltal i intervallet $1$ till $N$.

\section*{Indata}
Indatan av heltalet $N$ ($1 \leq N \leq 100$).

\section*{Utdata}
Skriv ut ett heltal: antalet lagomvinklade trianglar som uppfyller kriteriet.

\section*{Poängsättning}
Din lösning kommer att testas på en mängd testfallsgrupper.
För att få poäng för en grupp så måste du klara alla testfall i gruppen.

\noindent
\begin{tabular}{| l | l | p{12cm} |}
  \hline
  \textbf{Grupp} & \textbf{Poäng} & \textbf{Gränser} \\ \hline
  $1$    & $30$          & $N \leq 7$  \\ \hline
  $2$    & $70$          & Inga ytterligare begränsningar.  \\ \hline
\end{tabular}
