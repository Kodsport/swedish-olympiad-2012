\problemname{Talfamiljer}
Följande uppgift är en generalisering av ett problem från kvalet för Skolornas Matematiktävling hösten 2011. Vi säger att varje positivt heltal $N$ har en familj som består av $N$ samt alla positiva heltal man kan få genom att ordna om $N$:s siffror, utom dem som vid omordningen får en nolla som första siffra. (T.ex. har talet $101$ familjen $101$, $110$.) Vi säger också att $N$:s familj gillar det positiva heltalet $p$ om $N$ eller något annat tal i familjen är delbart med $p$. (Alla tal som familjen ovan gillar är $1,2,5,10,11,22,55,101,110$.)

Skriv ett program som, givet $N$ positiva heltal, bestämmer det minsta positiva heltalet vars familj gillar samtliga av de givna talen. I givna testfall kommer det alltid att finnas ett sådant tal med högst sex siffror.  
\section*{Indata}
Den första raden innehåller heltalet $N$ ($1\leq N \leq 10$). Sedan följer en rad med $N$ stycken positiva heltal, alla mindre än en miljon.

\section*{Utdata}
Skriv ut det minsta heltal vars familj gillar samtliga av de givna talen.

\section*{Poängsättning}
Din lösning kommer att testas på en mängd testfallsgrupper.
För att få poäng för en grupp så måste du klara alla testfall i gruppen.

\noindent
\begin{tabular}{| l | l | p{12cm} |}
  \hline
  \textbf{Grupp} & \textbf{Poäng} & \textbf{Gränser} \\ \hline
  $1$    & $40$          & Svaret är mindre än 100.  \\ \hline
  $2$    & $60$          & Inga ytterligare begränsningar.  \\ \hline
\end{tabular}
