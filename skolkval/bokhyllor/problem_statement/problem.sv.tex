\problemname{Bokhyllor}
Du ska köpa bokhyllor för att få plats med alla dina böcker.
Du vet från början vilka böcker du har och behöver räkna ut antalet bokhyllor som krävs.
Böckerna har tre olika storlekar: en liten bok tar $1$ platsenhet, en mellanstor $2$ och en stor bok tar $3$ platsenheter.
Varje hylla rymmer ett visst antal platsenheter.
Givet hur många böcker du har av varje sort, skriv ett program som beräknar hur många hyllor du behöver om du vill ha så få hyllor som möjligt.

\section*{Indata}
Den första raden innehåller tre heltal som, i tur och ordning, beskriver antalet små böcker, antalet mellanstora böcker samt antalet stora böcker.
Antalet böcker av varje sort kommer vara högst $20$.

Därefter följer en rad med heltalet $S$ ($S \le 20$), hyllstorleken.
Det kommer alltid krävas minst $2$ hyllor.

\section*{Utdata}
Skriv ut det minimala antalet hyllor som krävs för att få plats med alla böcker.

\section*{Poängsättning}
Ditt program kommer att testas på tre testfall.
Om du klarar två av dessa kommer du att få $50$ poäng.
Om du klarar alla tre kommer du att få $100$ poäng.

I två av testfallen är antalet böcker högst $25$ och hyllan har storlek högst $10$.
